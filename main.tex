\documentclass{report}
\usepackage[acronym,automake]{glossaries-extra}
\usepackage{hyperref}
\usepackage{unicode}
\usepackage{apacite}
\usepackage{parskip}
\usepackage{graphicx}
\usepackage{tabularx}
\usepackage{lipsum}

\setabbreviationstyle[acronym]{long-short}
\newacronym{startup}{fotogramas de inicio}{Cantidad de fotogramas para comenzar una movida}
\newacronym{active}{fotogramas activos}{Cantidad de fotogramas en la cual es posible colisionar con el oponente}
\newacronym{recovery}{fotogramas de recuperación}{Cantidad de fotogramas en la cual es imposible admitir otra movida}
\newacronym{frame_data}{datos fotográmicos}{los valores de fotogramas de inicio, fotogramas activos, fotogramas de recuperación y aturdimiento de bloqueo para una movida}
\newacronym{on_block}{bloqueando}{Cantidad de fotogramas de ventaja o desventaja luego de restar fotogramas de recuperación y aturdimiento de bloqueo}
\newacronym{blockstun}{aturdimiento de bloqueo}{Cantidad de fotogramas en que el atacado no puede hacer nada aparte de bloquear}
\newacronym{gbfvs}{GBFVS}{Granblue Fantasy Versus}
\makeglossaries

\title{SkyboundDB 2.0}
\author{Aramis Matos, Lenier Gerena}
\date{Primer Semestre, 2022-2033}

\begin{document}
\renewcommand{\acronymname}{Acrónimos}
\renewcommand{\bibname}{Referencias Bibliográficas}
\renewcommand{\contentsname}{Tabla de Contenido}
\renewcommand{\chaptername}{Capítulo}
\renewcommand{\figurename}{Figura}
\renewcommand{\tablename}{Tabla}

\maketitle

\tableofcontents

\chapter{Introducción}
\addcontentsline{toc}{chapter}{Introducción}

\section{Antecedentes}
\addcontentsline{toc}{section}{Antecedentes}

\section{Objetivos}
\addcontentsline{toc}{section}{Objetivos}

\subsection{Objetivo General}
\addcontentsline{toc}{subsection}{Objetivo General}

\subsection{Objetivos Específicos}
\addcontentsline{toc}{subsection}{Objetivos Específicos}

\section{Justificación}
\addcontentsline{toc}{section}{Justificación}

\section{Alcance}
\addcontentsline{toc}{section}{Alcance}




\chapter{Fundamentos Teóricos}
\addcontentsline{toc}{chapter}{Fundamentos Teóricos}

El propósito principal de un juego de pelea es la interacción entre dos personajes que mediante sus propias herramientas únicas puedan restar el recurso de su vida (o Health Points, HP \cite{noauthor_fighting_nodate} ) igual a 0. Bajo esta premisa sencilla, uno puede asumir que la mejor manera de llegar a dicha meta es usar todo movimiento que sea ofensivo y de mayor daño, pero dichas herramientas ofensivas pierden contra las herramientas defensivas. Ser defensivo no te lleva más cerca a la condición de ganar, sino que te mantiene sin perderla (mientras las herramientas se usen correctamente). Esto lleva a una interacción más compleja entre los jugadores, llevando el juego así a un extremo donde se quiere saber en qué momento es mejor tomar una por la otra para alcanzar el requisito de la victoria. Aquí entonces en donde importa el conocimiento de frame advantage relativamente a cada personaje. 

La mayoría de los juegos de pelea (y en todos los juegos de peleas modernos) corren consistentemente a 60 “frames per second”. Esto dicta qué tan bien se ve el juego cuando está corriendo, pues es la cantidad de imágenes que se procesan en dicho juego por segundo, pero también es una base principal para cómo funcionan las animaciones de dicho juego de pelea. Un “frame” se usa como una unidad básica de tiempo relativo (1 frame = 1/60 segundos) y es lo que permite que se pueda calcular el frame data de los personajes en el juego. Para propósitos competitivos, no se toma la idea de que se pueda reaccionar a lo que sería un frame, pues estarias reaccionando a algo que es imposible para la capacidad humana, sino se toma esta unidad para extraer situaciones de ventaja y desventaja en ciertas situaciones donde ambos jugadores presionan un botón al unísono (Fighting Game glossary, Infil, 2022).

Por consiguiente, es necesario conocer un poco acerca de la terminología que se utiliza en el espacio de los video juegos de pelea. Por lo general, los juegos de pelea se juegan con dos jugadores que se atacan con ciertas movidas. Estas movidas consumen un tiempo particular en completar. A continuación se presenta una lista definiendo términos importantes:
\begin{itemize}
    \item Movida que un jugador seleccionó (move)
    \item \gls{startup}
    \item \gls{active}
    \item \gls{recovery}
    \item \gls{blockstun}
    \item \gls{on_block}
\end{itemize}

\gls{frame_data} son particulares para cada movida de un personaje. A consecuencia de esto, es posible que algunas movidas sean mas rápidas que otras. Este dato es importante a la hora de calcular la ventaja que tiene una movida contra otra. 

Vamos asumir que hay una movida $A$ que tiene $25$ \gls{startup}, $3$ \gls{active}, $30$ \gls{recovery} y $-14$ \gls{on_block} y una movida $B$ que tiene $7$ \gls{startup}, $3$ \gls{active}, $6$ \gls{recovery} y $+3$ \gls{on_block}. Si el personaje de la movida $A$ ataca al personaje de la movida $B$ pero el personaje de la movida $B$ bloquea la movida, ahora el personaje $A$ experimenta \gls{recovery}. Ahora, el personaje de la movida $A$ no puede hacer nada por $30$ frames. Al solo tener $16$ frames de \gls{blockstun}, el personaje de la movida $A$ experimenta $14$ que no puede hacer nada pero puede ser atacado. A este estado se le llama desventaja. Al mismo tiempo, el personaje de la movida $B$ tiene 14 frames para hacer lo que quiera. Como su movida solo se tarda $7$ frames en salir, puede atacar al personaje de la movida $A$ sin miedo de ser atacado.

\chapter{Aspectos Humanos}
\addcontentsline{toc}{chapter}{Aspetos Humanos}

\begin{tabularx}{\textwidth}{| X | X | X |}
    \hline \textbf{Usuario} & \textbf{Clasificación} & \textbf{Circunstancias de Uso} \\
    \hline Jugadores de \gls{gbfvs} & Jugadores Novatos \newline Jugadores Expertos & Antes, luego y entre partidas \newline El tiempo entre partidas es bien corta (1-3 minutos) \newline  En torneos competitivos \\
    \hline
\end{tabularx}

\begin{tabularx}{\textwidth}{| X | X | X |}
    \hline \textbf{Elemento} & \textbf{Descripción} & \textbf{Importancia} \\
    \hline \textbf{Datos básicos} & Usuarios: Jugadores de \gls{gbfvs} \newline Limitaciones: No se puede comparar todas las movidas de un personaje con todas de otro  & Si un usuario no es jugador de \gls{gbfvs}, no le va a conseguir utilidad a la aplicación\\
    \hline \textbf{Características físicas} & La audiencia para la aplicación es aquellos de 13 años o mas, independiente de genero o sexo. Si el usuario puede navegar el internet, puede acceder la aplicación & \gls{gbfvs}, como muchos juegos de pelea, son bien accesibles a personas con discapacidades. Es por esto, se piensa seguir las guías de accesibilidad para la \textit{web}\\ 
    \hline \textbf{Características psicológicas} & Se asume que el usuario es alfabeta, conoce la notación de teclado numérico \cite{noauthor_numpad_nodate} y conoce acerca de \gls{frame_data} & Debido a la manera que se extrae la data de su fuente de origen, el usuario debe ser capaz de comprender la notación de teclado numérico\\
    \hline \textbf{Dispositivos comúnmente usados} & La aplicación es una página \textit{web}, por ende, cualquier dispositivo que pueda navegar el internet, puede acceder la aplicación.  & La mayoría de la demografía de la aplicación tienen acceso a un dispositivo móvil a todo momento. \\
\end{tabularx}

\begin{tabularx}{\textwidth}{| X |  X |  X |}
    & Sin embargo, la aplicación está diseñada principalmente para dispositivos móviles y computadoras & Este punto es importante porque la aplicación es útil cuando se esté compitiendo\\
    \hline \textbf{Modelo mental del sistema} & El modelo mental en la cual se basa la aplicación es la del menú de selección de personaje como se presenta en las figuras \ref{fig: strive chracter select} y \ref{fig: melty character select}, redondeadas por un rectángulo azul & El adaptarse a un modelo mental pre-existente facilita el aprendizaje del programa y por ende, acelera la adquisición de pericia y limita la frustración con la aplicación  \\ 
    \hline \textbf{Metas} & Al iniciar, se le ofrece al usuario las entradas de el el personaje que inició el ataque, el que respondió al ataque y sus respectivas movidas de cada personaje seleccionado. Al llegar a la página de resultados, el usuario logro seleccionar el personaje que inició el ataque, el que respondió al ataque y sus respectivas movidas y logró ver como son los datos fotográmicos de las movidas al ser comparadas & Dado que la naturaleza de la aplicación es de calculadora fotográmica, es obvio entonces que es de gran importancia que el usuario consiga la información que busca \\ 
    \hline \textbf{Requisitos} & Se piensa que la aplicación sera utilizada espontáneamente entre o antes partidas. Es por esto que aplicación debe ser sencilla y rápida de acceder & La rapidez y sencillez y sumamente importante debido a que el tiempo entre partidas en un torneo es bien corto (entre 1-3 minutos). Se tiene que tener acceso inmediato a la información deseada\\ 
    \hline
\end{tabularx}

\textbf{Metáfora de la interfaz gráfica}: Los juegos de pelea generalmente tienen un diseño de similar a dos cajas verticales, como se puede apreciar en la figura \ref{fig: abstract model}:

\begin{figure}[ht!]
    \centering
    \caption{Modelo abstracto del menú de los juegos de pelea}
    \includegraphics[height=0.2\textwidth]{figures/abstract-design.jpg}
    \label{fig: abstract model}
\end{figure}

Dentro de estas dos cajas, están los personajes que se han seleccionado y su color de traje, como se puede ver en la figura \ref{fig: strive chracter select}:

\begin{figure}[ht!]
    \centering
    \caption{Menú de selección de personaje de Guilty Gear Strive}
    \includegraphics[height=0.5\textwidth]{figures/guilty.jpg}
    \label{fig: strive chracter select}
\end{figure}

Lo mismo ocurre en otros juegos como Melty Blood: Type Lumina \cite{noauthor_melty_nodate}, la figura \ref{fig: melty character select}:

\begin{figure}[ht!]
    \centering
    \caption{Menú de selección de personaje de Melty Blood}
    \includegraphics[height=0.5\textwidth]{figures/melty.jpg}
    \label{fig: melty character select}
\end{figure}

Como se ve en ambos ejemplos, los personajes seleccionados están colocados en esquinas opuestas, cada uno en su propia caja. El único elemento que se comparte entre los dos es la lista de personajes (caja azul)

\textbf{Características de la interfaz gráfica}: 

\begin{itemize}
    \item \textbf{Aspectos ergonómicos}
    \begin{itemize}
        \item El texto dentro de la aplicación tiene que ser legible
        \item Los menús desplegables tienen que ser lo suficientemente grande
        \item Se tiene que seguir los estándares de accesibilidad de HTML, tales como el uso de atributos \textit{alt} y el uso apropiado, lógico y consistente de etiquetas
        \item El botón de someter tiene que ser lo suficientemente claro en su propósito y suficientemente grande para que sea cómodo para todo tipo de usuario
    \end{itemize}
    \item \textbf{Colores}
    
    La paleta de colores es la siguiente:
    \begin{figure}[ht!]
        \centering
        \caption{Paleta de Colores}
        \includegraphics[width=1.0\textwidth]{figures/pallete_updated.png}
        \label{fig: pallete}
    \end{figure}

    Se utiliza el color del medio (\#3385D6 en hexadecimal) como color de fondo. Esto se debe a que al ser azul, distrae al usuario menos que un color como el rojo. Más aún, el color temático de \gls{gbfvs} es el azul claro. Esta relación ayuda a familiarizar el usuario con que juego esta trabajando esta aplicación. El color que se usa para los bordes es el color complementario de \#3385D6, el complemento de \#D69933 (marrón claro). Al marrón ser una mezcla del rojo y el verde, es bueno para detalles y para segmentar elementos de una manera visual.
    \item \textbf{Interactividad}
    \begin{itemize}
        \item Completa
        \begin{itemize}
            \item Imágenes
            \item Apuntador
            \item Teclado
        \end{itemize}
        \item Parcial
        \begin{itemize}
            \item Teclado
            \item Auditivo
        \end{itemize}
    \end{itemize}
    \item \textbf{Validación de datos}
    
    Los fotogramas de cada personaje provienen de \textit{Dustloop} \cite{noauthor_granblue_2022-1}. Estos datos se extraen cada cierto periodo de tiempo de \textit{Dustloop} automáticamente. Se asume que debido a que \gls{gbfvs} ya no está recibiendo actualizaciones que \textit{Dustloop} no cambiará drásticamente la estructura de las páginas de donde se extrae la data. Por ende, se asume que la data fotográmica sera confiable por un buen tiempo.

    Debido a la naturaleza de los menús desplegables, no hay mucha validación necesaria en cuanto a las opciones que se le presentan al usuario. Sin embargo, hay un detalle importante que si se tiene que validar. La aplicación original de \textit{SkyboundDB} tiene la opción de comparar una movida con todas las movidas de otro personaje. Esto implica que no es posible comparar todas las movidas de un personaje con todas de otro por cuestiones de limpieza y por cuestiones prácticas debido a la cantidad de información que se tiene que presentar. A consecuencia de esto, se tiene que validar que el personaje que dio el primer golpe no tenga la opción de \textbf{Seleccionar todas las movidas} seleccionada y que el personaje que responde tampoco lo tenga a la vez. Solo el personaje que responde puede utilizar la opción de seleccionar todas sus movidas. Esta validación se tiene que hacer antes que se haga la comparación.
    % \item \textbf{Consistencia}
    \item \textbf{Carga de memoria}
    
    % Idealmente, la carga de memoria debe ser baja por varias razones: Unas de las metas primordiales de este proyecto es que la aplicación sea rápida y eficiente. Unas de las métricas de eficiencia es la memoria primaria necesaria para mantener la página abierta. Otra métrica importante es la cantidad de banda ancha necesaria para acceder la página en linea. Unos de los casos de usos que tiene la aplicación es su uso en torneos, particularmente entre o antes de partidas. Esta circunstancia requiere que la pagina cargue rápidamente y sin tardarse mucho las comparaciones.  

    En la aplicación original de SkyboundDB, cada personaje tenía su propia foto como ícono para seleccionarlo. Esto hacia que la interfaz sea mas interactiva e intuitiva. Esta decisión era viable porque solo habían 6 personajes, no se añadieron todos los 28 personajes por cuestiones de tiempo. SkyboundDB 2.0 tendrá la opciones se seleccionar cualquier de los 28 personajes. Es por esto que no se puede tener un ícono por cada personaje. Es intimidante tener tanta opciones. Es muy común ver a un jugador novato de juegos de pelea quedarse paralizado al ver cuantos personajes puede usar. Es por esto se tiene que conseguir una manera de colapsar todas las opciones de personajes en un solo elemento, aunque se tenga que sacrificar interactividad.

    \item \textbf{Indicaciones visuales}
    
    Debido a las limitaciones de banda ancha, la página no puede tener una plantilla de estilo demasiada complicada. Sin embargo, elementos sencillos pueden mejorar la experiencia de usuario sin sacrificar mucho en términos de rapidez. Una imagen simple del personaje que se ha seleccionado, como se presenta en la figura \ref{fig: melty character select}, puede ayudar corroborar rápidamente si la selección del usuario fue la que quiso. Además, se puede estilizar la selección de movida para que se asemeje mas a la selección de traje como se puede apreciar con el dígito $2$ con las flechas en la parte derecha de las figuras \ref{fig: strive chracter select} y \ref{fig: melty character select}. Estos detalles ayudan a conformarse al modelo mental que tiene el usuario de los juegos de pelea.
\end{itemize}

\chapter{Aspectos Tecnológicos}
\addcontentsline{toc}{chapter}{Aspectos Tecnológicos}

\begin{center}
    \begin{tabular}{|c|c|c|}
        \hline
        \textbf{Tipo de dispositivos} & \textbf{Descripción} & \textbf{Ejemplos} \\
        \hline
        \textbf{Entrada} & & \\
        \hline
        \textbf{Salida} & & \\
        \hline
        \textbf{Entrada/Salida} & & \\
        \hline
    \end{tabular}
\end{center}

\chapter{Diseño de la interfaz gráfica de usuario}
\addcontentsline{toc}{chapter}{Diseño de la interfaz gráfica de usuario}

\section{Diseño de la interfaz gráfica de usuario}
\addcontentsline{toc}{section}{Diseño de la interfaz gráfica de usuario}
\textbf{Análisis de aplicaciones similares}: No es posible mejorar un producto sin analizarlo y criticarlo. Por ende, es necesario comparar el diseño del prototipo del interfaz gráfica de usuario para así mejorarlo en términos de usabilidad y comodidad.

Las aplicaciones que se van a utilizar como objectos de comparación son las mismas que se utilizaron los antecedentes, \textit{FAT - Frame Data!} y \textit{Smash Ultimate Calculator}.

Como se puede apreciar en la figuras \ref{fig: char sel}, \ref{fig: game select}, \ref{fig: compare options} y \ref{fig: comparison}, \textit{FAT - FRAME DATA!} parece utilizar diseño material \cite{noauthor_designing_nodate}. Mas aún, en la figura \ref{fig: char sel}, se utilizan las siluetas de los personajes con el color predominante del personaje como color de fondo. Esto ayuda reducir la carga de memoria de los usuarios mientras que no sacrifica mucho en términos de usabilidad. Es una buena decisión de diseño y es algo que falta en la interfaz gráfica de usuario de SkyboundDB 2.0. Al utilizar principios del diseño material, \textit{FAT - Frame Data!} es inmediatamente familiar a usuarios de aplicaciones de \textit{Google} y \textit{Android}. El uso de diseño material fue una buena decisión de diseño ya que facilita la adaptación de los usuarios a una aplicación nueva porque comparte principios y estándares de diseño.

\begin{figure}
    \centering
    \includegraphics[height=0.4\textheight]{figures/char_sel.png}
    \caption{Selección de personaje en \textit{FAT - FRAME DATA!}}
    \label{fig: char sel}
\end{figure}

\begin{figure}
    \centering
    \includegraphics[height=0.4\textheight]{figures/game_options.png}
    \caption{Selección de juego en \textit{FAT - FRAME DATA!}}
    \label{fig: game select}
\end{figure}

\begin{figure}
    \centering
    \includegraphics[height=0.4\textheight]{figures/compare_options.png}
    \caption{Opciones de comparación en \textit{FAT - FRAME DATA!}}
    \label{fig: compare options}
\end{figure}

\begin{figure}
    \centering
    \includegraphics[height=0.4\textheight]{figures/comparison.png}
    \caption{Comparación entre movidas en \textit{FAT - FRAME DATA!}}
    \label{fig: comparison}
\end{figure}

\newpage

\textit{Smash Ultimate Calculator} no parece utilizar un estándar de diseño al igual que SkyboundDB 2.0. Esto crea confusión al utilizar la aplicación por primera vez. Como se puede ver en las figuras \ref{fig: SUC main menu} y \ref{fig: SUC additional options}, a pesar que las areas que contienen información están agrupadas lógicamente, hay demasiados elementos y opciones que se presentan al usuario. Esto incurre una gran carga de memoria e intimida al usuario novato. También es difícil ver que cambios ocurren al oprimir opciones. A pesar de estas decisiones de diseño problemáticas, la paleta de colores es pasable, sino un poco deprimente y sin vida.

\begin{figure}[ht!]
    \centering
    \includegraphics[width=0.75\textwidth]{figures/SUC1.png}
    \caption{Menú principal de \textit{Smash Ultimate Calculator}}
    \label{fig: SUC main menu}
\end{figure}

\begin{figure}[ht!]
    \centering
    \includegraphics[width=0.75\textwidth]{figures/SUC2.png}
    \caption{Opciones adicionales de \textit{Smash Ultimate Calculator}}
    \label{fig: SUC additional options}
\end{figure}

\newpage

\textbf{Definición de Aspectos de la GUI}: 

\begin{itemize}
    \item Objetos
    \item Acciones de la interfaz gráfica de usuario
    \item Iconos
    \item Vistas
    \item Representaciones visuales de los Objetos
    \item Menú de los Objetos
    \item Ventanas
\end{itemize}

\textbf{Estructura de la interfaz gráfica de usuario}:

\section{Prototipo}
\addcontentsline{toc}{section}{Prototipo}

\href{https://aramis-matos.github.io/skyboundDB2.0/}{SkyboundDB 2.0}

\printglossary[type=\acronymtype]

\bibliography{References}

\bibliographystyle{apacite}

\end{document}
