\documentclass{report}
\usepackage{apacite}
\usepackage{parskip}
\usepackage[utf8]{inputenc}
\usepackage{graphicx}
\usepackage[acronym,automake]{glossaries-extra}
\setabbreviationstyle[acronym]{long-short}
\newacronym{startup}{startup frames}{Cantidad de frames para comenzar una movida}
\newacronym{active}{active frames}{Cantidad de frames en la cual es posible colisionar con el opnente}
\newacronym{recovery}{recovery frames}{Cantidad de frames en la cual es imposible admitir otra movida}
\newacronym{frame_data}{frame data}{Los valores de startup frames, active frames, recovery frames y blockstun para una movida}
\newacronym{on_block}{on block}{Cantidad de frames de ventaja o desventaja luego de restar recovery frames y blockstun}
\newacronym{blockstun}{blockstun}{Cantidad de frames en que el atacado no puede hacer nada aparte de bloquear}
\makeglossaries

\title{COMP4998 Reporte}
\author{Aramis Matos, Lenier Gerena, Jorge Huertas}
\date{Primer Semestre, 2022-2033}

\begin{document}

\maketitle

\tableofcontents

\chapter{Introducción}
\addcontentsline{toc}{chapter}{Introducción}

\section{Antecedentes}
\addcontentsline{toc}{section}{Antecedentes}

\section{Objetivos}
\addcontentsline{toc}{section}{Objetivos}

\subsection{Objetivo General}
\addcontentsline{toc}{subsection}{Objetivo General}

\subsection{Objetivos Específicos}
\addcontentsline{toc}{subsection}{Objetivos Específicos}

\section{Justificación}
\addcontentsline{toc}{section}{Justificación}

\section{Alcance}
\addcontentsline{toc}{section}{Alcance}




\chapter{Fundamentos Teóricos}
\addcontentsline{toc}{chapter}{Fundamentos Teóricos}

El propósito principal de un juego de pelea es la interacción entre dos personajes que mediante sus propias herramientas únicas puedan restar el recurso de su vida (o Health Points, HP \cite{noauthor_fighting_nodate} ) igual a 0. Bajo esta premisa sencilla, uno puede asumir que la mejor manera de llegar a dicha meta es usar todo movimiento que sea ofensivo y de mayor daño, pero dichas herramientas ofensivas pierden contra las herramientas defensivas. Ser defensivo no te lleva más cerca a la condición de ganar, sino que te mantiene sin perderla (mientras las herramientas se usen correctamente). Esto lleva a una interacción más compleja entre los jugadores, llevando el juego así a un extremo donde se quiere saber en qué momento es mejor tomar una por la otra para alcanzar el requisito de la victoria. Aquí entonces en donde importa el conocimiento de frame advantage relativamente a cada personaje. 

La mayoría de los juegos de pelea (y en todos los juegos de peleas modernos) corren consistentemente a 60 “frames per second”. Esto dicta qué tan bien se ve el juego cuando está corriendo, pues es la cantidad de imágenes que se procesan en dicho juego por segundo, pero también es una base principal para cómo funcionan las animaciones de dicho juego de pelea. Un “frame” se usa como una unidad básica de tiempo relativo (1 frame = 1/60 segundos) y es lo que permite que se pueda calcular el frame data de los personajes en el juego. Para propósitos competitivos, no se toma la idea de que se pueda reaccionar a lo que sería un frame, pues estarias reaccionando a algo que es imposible para la capacidad humana, sino se toma esta unidad para extraer situaciones de ventaja y desventaja en ciertas situaciones donde ambos jugadores presionan un botón al unísono (Fighting Game glossary, Infil, 2022).

Por consiguiente, es necesario conocer un poco acerca de la terminología que se utiliza en el espacio de los video juegos de pelea. Por lo general, los juegos de pelea se juegan con dos jugadores que se atacan con ciertas movidas. Estas movidas consumen un tiempo particular en completar. A continuación se presenta una lista definiendo términos importantes:
\begin{itemize}
    \item Movida que un jugador seleccionó (move)
    \item \gls{startup}
    \item \gls{active}
    \item \gls{recovery}
    \item \gls{blockstun}
    \item \gls{on_block}
\end{itemize}

\gls{frame_data} son particulares para cada movida de un personaje. A consecuencia de esto, es posible que algunas movidas sean mas rápidas que otras. Este dato es importante a la hora de calcular la ventaja que tiene una movida contra otra. 

Vamos asumir que hay una movida $A$ que tiene $25$ \gls{startup}, $3$ \gls{active}, $30$ \gls{recovery} y $-14$ \gls{on_block} y una movida $B$ que tiene $7$ \gls{startup}, $3$ \gls{active}, $6$ \gls{recovery} y $+3$ \gls{on_block}. Si el personaje de la movida $A$ ataca al personaje de la movida $B$ pero el personaje de la movida $B$ bloquea la movida, ahora el personaje $A$ experimenta \gls{recovery}. Ahora, el personaje de la movida $A$ no puede hacer nada por $30$ frames. Al solo tener $16$ frames de \gls{blockstun}, el personaje de la movida $A$ experimenta $14$ que no puede hacer nada pero puede ser atacado. A este estado se le llama desventaja. Al mismo tiempo, el personaje de la movida $B$ tiene 14 frames para hacer lo que quiera. Como su movida solo se tarda $7$ frames en salir, puede atacar al personaje de la movida $A$ sin miedo de ser atacado.

\printglossary[type=\acronymtype]

\bibliography{References}

\bibliographystyle{apacite}

\end{document}
