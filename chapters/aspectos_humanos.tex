
\begin{tabularx}{\textwidth}{| X | X | X |}
    \hline \textbf{Elemento} & \textbf{Descripción} & \textbf{Importancia} \\
    \hline \textbf{Datos básicos} & Usuarios: Jugadores de \gls{gbfvs} \newline Limitaciones: No se puede comparar todas las movidas de un personaje con todas de otro  & Si un usuario no es jugador de \gls{gbfvs}, no le va a conseguir utilidad a la aplicación\\
    \hline \textbf{Características físicas} & La audiencia para la aplicación es aquellos de 13 años o mas, independiente de genero o sexo. Si el usuario puede navegar el internet, puede acceder la aplicación & \gls{gbfvs}, como muchos juegos de pelea, son bien accesibles a personas con discapacidades. Es por esto, se piensa seguir las guías de accesibilidad para la \textit{web}\\ 
    \hline \textbf{Características psicológicas} & Se asume que el usuario es alfabeta, conoce la notación de teclado numérico \cite{noauthor_numpad_nodate} y conoce acerca de \gls{frame_data} & Debido a la manera que se extrae la data de su fuente de origen, el usuario debe ser capaz de comprender la notación de teclado numérico\\
    \hline \textbf{Dispositivos comúnmente usados} & La aplicación es una página \textit{web}, por ende, cualquier dispositivo que pueda navegar el internet, puede acceder la aplicación.  & La mayoría de la demografía de la aplicación tienen acceso a un dispositivo móvil a todo momento. \\
\end{tabularx}

\begin{tabularx}{\textwidth}{| X |  X |  X |}
    & Sin embargo, la aplicación está diseñada principalmente para dispositivos móviles y computadoras & Este punto es importante porque la aplicación es útil cuando se esté compitiendo\\
    \hline \textbf{Modelo mental del sistema} &  & \\ 
    \hline \textbf{Metas} & Al iniciar, se le ofrece al usuario las entradas de el el personaje que inició el ataque, el que respondió al ataque y sus respectivas movidas de cada personaje seleccionado. Al llegar a la página de resultados, el usuario logro seleccionar el personaje que inició el ataque, el que respondió al ataque y sus respectivas movidas y logró ver como son los datos fotográmicos de las movidas al ser comparadas & Dado que la naturaleza de la aplicación es de calculadora fotográmica, es obvio entonces que es de gran importancia que el usuario consiga la información que busca \\ 
    \hline \textbf{Requisitos} & Se piensa que la aplicación sera utilizada espontáneamente entre o antes partidas. Es por esto que aplicación debe ser sencilla y rápida de acceder & La rapidez y sencillez y sumamente importante debido a que el tiempo entre partidas en un torneo es bien corto (entre 1-3 minutos). Se tiene que tener acceso inmediato a la información deseada\\ 
    \hline
\end{tabularx}

