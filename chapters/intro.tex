\section{Antecedentes}
\addcontentsline{toc}{section}{Antecedentes}

El cálculo de ventaja de "frame data", por lo general, ha sido una tarea que se hace a mano \cite{noauthor_guide_nodate-1} y \cite{dustloop_using_2022}. Esto se debe a que los números involucrados son pequeños por lo general y por ende, no se ha necesitado una gran cantidad de recursos computacionales. Sin embargo, con la venida de  torneos de video juegos competitivos \cite{willingham_what_2018-1}, se ha visto una necesidad de conocer estados de ventaja rápida y efectivamente. 

\href{https://qph.cf2.quoracdn.net/main-qimg-35650aa12d02e1700face4fbf71f4cfd}{HADOOKEN} 
\cite{noauthor_grasshopper_nodate-1} 

Por la sencillez de los cálculos, no se ha innovado mucho en el espacio de calculadoras de frame data. Mas aún, las diferencias en mecánicas internas entre juegos ha complicado el asunto de una calculadora generalizada pero han habido algunos intentos para crear una, entre ellas están:
\begin{itemize}
    \item FAT - Frame Data! \cite{d4rkonion_fat_2022-1}
    \item Smash Ultimate Calculator \cite{noauthor_rubendalssbu-calculator_nodate-1}
\end{itemize}

FAT - Frame Data! es una aplicación móvil que provee una gran cantidad de estadísticas para los juegos Street Fighter III: 3rd Strike \cite{noauthor_street_2022-5}, Ultra Street Fighter IV \cite{noauthor_street_2022-4}, Street Fighter V \cite{noauthor_street_2022-3} y Guilty Gear Strive \cite{noauthor_guilty_2022-1}. Entre sus funcionalidades están tablas completas de \gls{frame_data}, búsqueda de \gls{frame_data} mediante el nombre del personaje y una movida, listas completas de movidas y su entrada correspondiente en el control, comparación de estadísticas internas de cada personaje y una calculadora de \gls{frame_data}. Sin embargo, la calculadora no calcula estados de ventaja, sino para calcular si es posible que el oponente ataque entre dos movidas si la primera es bloqueada. No se calcula estados de ventaja y desventaja. En términos de interfaz de usuario, la aplicación no se ve mal estéticamente. Utiliza el negro como color de fondo y el azul para detalles. La aplicación presenta la información claramente y segmentada lógicamente para así no intimidar el usuario. Entre todo, es un interfaz bien hecho.

Smash Ultimate Calculator es una aplicación web que calcula la cantidad de daño que sufre el oponente y la cantidad de daño necesaria para que el oponente pierda una vida en el juego Super Smash Bros. Ultimate \cite{noauthor_super_2022-1}. Es notable que esta aplicación solo se puede aplicar al juego de Super Smash Bros. Ultimate, ningún otro. Hay varias razones por la que esto es así. Primero, Super Smash Bros. Ultimate no es un juego de pelea tradicional ya que sus condiciones de victoria y manera de mover el personaje son radicalmente distintas a juegos como Street Fighter V y Guilty Gear Strive. Segundo, Super Smash Bros. Ultimate tiene unos mecanismos que divergen lo suficiente para que no apliquen las reglas de los juegos anteriores en su serie como Super Smash Bros. Melee \cite{noauthor_super_nodate-1}. Es por estas razones que Smash Ultimate Calculator no tiene aplicabilidad fuera del juego de la cual fue diseñado. Aparte de esto, la interfaz se ve primitiva. Muchos de los items en la páginas se ven dispersos y sin clara definición de donde elementos relacionados empiezan y terminan. A la misma vez, el interfaz se ve demasiado ocupado como que intimida y confunde al usuario.

En conclusión, a pesar de algunos buenos intentos como FAT - Frame Data!, no hay muchas aplicaciones que presenten las estadísticas de estado de ventaja y desventaja de manera limpia y concisa.

\section{Objetivos}
\addcontentsline{toc}{section}{Objetivos}

\subsection{Objetivo General} 
\addcontentsline{toc}{subsection}{Objetivo General}

El grupo desea crear un interfaz mas amigable y atractivo para una aplicación de cálculo de estado de ventaja para juegos de pelea.

\subsection{Objetivos Específicos}
\addcontentsline{toc}{subsection}{Objetivos Específicos}

El grupo tiene como objetivo tres metas:
\begin{itemize}
    \item Rehacer y mejorar el interfaz de la aplicación SkyboundDB \cite{aramis_matos_aramis-matosskybounddb_2021-1}
    \item Volver a implementar el interfaz de la aplicación en HTML y CSS
    \item Si es possible, volver a implementar la aplicación en JavaScript en vez de Python
\end{itemize}

\section{Justificación}
\addcontentsline{toc}{section}{Justificación}
SkyboundDB se creó originalmente como un proyecto final para otro curso. Sin embargo, las razones por las cuales se decidió crear el proyecto original fueron varias. Uno de autores principales de SkyboundDB, Lenier Gerena, siempre ha sido un jugador ávido de los juegos de pelea y participante activo en torneos competitivos de tales juegos. Siempre ha querido de contribuir algo a la comunidad que le ha ofrecido tantas horas de entretenimiento y un sinnúmero de amistades. Este deseo se combinó con el deseo del segundo autor principal, Aramis Matos, de crear una aplicación gráfica de usuario. Por esto, SkyboundDB resultó ser una aplicación con GUI que calcula estados de ventaja y desventaja para algunos personajes del juego Granblue Fantasy Versus \cite{noauthor_granblue_2022-1}.

SkyboundDB
se creó con el propósito de simplificar el proceso de experimentación necesario para ver que movidas se pueden usar para castigar a otras. Por lo general, esto es un proceso iterativo que requiere paciencia e intuición debido a la gran cantidad de posibles combinaciones y variables. Esto es un proceso frustrante que realmente desilusiona al principiante de los juegos de pelea. Esta es otra razón por la cual se desarrolló SkyboundDB.

\section{Alcance y Limitaciones}
\addcontentsline{toc}{section}{Alcance y Limitaciones}

Se piensa rehacer el interfaz de las 4 pantallas principales que tiene SkyboundDB:
\begin{itemize}
    \item Selección de personajes (pantalla principal)
    \item Selección de movida de un personaje
    \item Comparación de una movida con otra
    \item Comparación de una movida con todas otras movidas de otro personaje
\end{itemize}

Además, si es posible, se gustaría volver a implementar el programa en HTML, CSS y JavaScript para así tener un prototipo funcional.

En términos de limitaciones, no se piensa expandir al la cantidad secciones que tiene el programa por cuestiones de tiempo. Solo se creará el prototipo funcional si es posible dado el tiempo, no es una garantía que se valla hacer.


