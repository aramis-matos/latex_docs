\begin{tabularx}{\textwidth}{| X | X | X |}
    \hline
    \textbf{Tipo de dispositivos} & \textbf{Descripción} & \textbf{Ejemplos} \\
    \hline
    \textbf{Entrada} & Los usuarios entran la data mediante un dispositivo apuntador en las cajas de seleccion multiples las movidas de tanto el personaje que hace la movida y el personaje que recibe la movida & Ratón, teclado y lápiz óptico como medios de entrada. \\
    \hline
    \textbf{Salida} & Se proyecta los resultados de la comparación mediante una pantalla o mediante ayuda de sonido por parte de asistencia de una aplicación lectora. & Monitor, altavoz, bocinas y auriculares como medios de salida. \\
    \hline
    \textbf{Entrada/Salida} & Los usuarios lo usan para tanto entrar la información mediante las cajas de selección multiples y a su vez se le presenta la información de las comparaciones mediante el mismo dispositivo. & Pantalla táctil. (teléfono y tablet). \\
    \hline
\end{tabularx}

\textbf{Estilos de Interacción}: Se toma en mente que este programa solo se puede accesar desde el internet, pues es una pagina web. Bajo esta lógica, se sigue que, por razones de ancho de banda, la pagina no debe tener mucho estilo (como imagenes de todas las movidas e imagenes decorativas) para asi no tomar tiempo cargando dicha pagina y substraendo de su practicalidad. Por eso la pagina se hace todo dentro de un mismo documento HTML que se alimenta de un CSS y un JS, de manera que en todo momento la información necesaria esta viva y no tiene que cargar nada en adicional. Esto deja que la mayoría del tiempo que un usuario este interactuando con la pagina terminará siendo el tiempo que tomen en escojer los personajes y sus movidas. Esto tambien se ha facilitado, pues las listas de personajes se organizaron de manera alfabeticas y esto permite que el usuario pueda navegar por rangos en el teclado presionando la primera letra del nombre del personaje que se esta hallando. 

También se presenta los resultados como listados tabulados con colores para saber qué tipo de movimiento es (si es un boton liviando-rosa, mediano-verde, o pesado-rojo) para dejarle saber al usuario sin necesitar imagenes y de la manera más simple pero organizada los resultados de dicha situación proveída por el usuario.

\textbf{Características de la Interfaz Gráfica}: Las opciones se podran accesar mediante un dispositivo apuntador que el usuario decida usar. Esto entonces nos deja con que el usuario nominalmente puede usar cualquier dispositivo que tenga capacidad de usar un navegador web y algun tipo de dispositivo apuntador para interactuar con ella, sea un lapiz optico en una tablet o (el comunmente usado) ratón. Cómo manera adicional de acceder la aplicación en la web, los usuarios pueden someter la información atravez de una pantalla tactil que funciona tanto como un periferio de entrada de la información, asi como la salida (se presenta los resultados en la misma pantalla). Coonsecuentemente, al ser una aplicación web, no habra muchas otras maneras de interactuar con ella más que con las cajas de selección y por aplicaciones de texto a altavoz mediante los textos alternativos proveidos para justamente esa necesidad (de faltar ayuda auditoria en vez de visión).